\documentclass{article}
\usepackage{lastpage}
\usepackage{color}
\usepackage[utf8]{inputenc}
\usepackage{multicol}
\setlength{\columnsep}{1cm}
\usepackage{graphicx}
\usepackage{wrapfig}
\usepackage{array}
\usepackage{fancyhdr}
\usepackage{hyperref}
\usepackage[a4paper,margin=1in]{geometry}
\usepackage{etoolbox}
\usepackage{listings}
\definecolor{purple}{rgb}{0.65, 0.12, 0.82}
\definecolor{lightgray}{rgb}{.5,.5,.5}

\makeatletter
\patchcmd{\@maketitle}{\LARGE}{\fontsize{14}{37}\selectfont}{}{}
\makeatother

\lstdefinelanguage{JavaScript}{
  keywords={typeof, new, true, false, catch, function, return, null, catch, switch, var, let, if, in, while, do, else, case, break},
  keywordstyle=\color{blue}\bfseries,
  ndkeywords={class, export, boolean, throw, async, await, implements, import, this},
  ndkeywordstyle=\color{red}\bfseries,
  identifierstyle=\color{black},
  sensitive=false,
  comment=[l]{//},
  morecomment=[s]{/*}{*/},
  commentstyle=\color{lightgray}\ttfamily,
  stringstyle=\color{red}\ttfamily,
  morestring=[b]',
  morestring=[b]"
}
\title{Compte rendu du TP1 d'INFO 834}
\author{Maxime PAULIN \and Ryan RISS}

\lstset{
  breaklines=true,
}

\hypersetup{
	colorlinks=true,
	linkcolor=blue,
	filecolor=magenta,      
	urlcolor=cyan,
	pdftitle={Overleaf Example},
	pdfpagemode=FullScreen,
}


\pagestyle{fancy}
\renewcommand{\headrulewidth}{1pt}

\fancyhead[R]{\includegraphics[height=1.5cm]{logoUSMB}}
\fancyhead[L]{\includegraphics[height=1.5cm]{logoPolytech}}

\renewcommand{\footrulewidth}{2pt}


\fancyfoot[C]{\thepage}

\geometry{
  a4paper,
  left=30mm,
  right=30mm,
  headheight=4cm,
  top=5.5cm,
  bottom=2.5cm
}

\begin{document}

\graphicspath{}
\maketitle
\begin{center}
	\includegraphics[height=5cm]{logoPolytech}
\end{center}
\newpage

%\section*{Abstract}
%\input{abstract}
%\newpage

%\tableofcontents

\newpage

\section*{Préface}
Nous avons installé mongodb sur un vps que nous possèdons pour que cela soit 
plus pratique a travailler en groupe de 2. Le serveur est situé à Torronto, 
ce qui ajoute ~100ms de ping par requête.
\setcounter{section}{5}
\section{Curseurs}
\begin{lstlisting}[language=javascript]
let q6 = async () => {
	let cursor = collections.communes.find({});
	let timeJson = {};
	let time = Date.now();
	while(await cursor.hasNext()){
		console.log(await cursor.next());
	}
}
\end{lstlisting}
On récupère l'intégralité de la base de donnée. C'est un peu long, mais ca marche.

\section{Benchmark}
\begin{lstlisting}[language=javascript]
let q7 = async () => {
	//find all towns
	let townList = [];
	let cursor = collections.communes.find({});
	while(await cursor.hasNext()){
		townList.push((await cursor.next()).nom_commune);
	}

	let timeJson = {};//for usefull data holder
	let timeList = [];//for handy time calc
	for(let town of townList){
		let time = Date.now();

		if(timeList.length > 2000) break;
		//150ms * 39000 ~ 2hours, cant be bothered
		await collections.communes.findOne({nom_commune: town})
		timeJson[town] = Date.now() - time;
		timeList.push(timeJson[town]);
		console.log(town, ":", timeJson[town])
		}
		
	let total = 0;
	for(let k of timeList) {//calc of average
			total += k;
	}
	console.log("Average time : ", total / timeList.length);
	//140ms, but my server is hosted in Torronto (approx 110ms ping)
	//110 when adding an index to the db
	//after indexing : 112 ms
}
\end{lstlisting}

On trouve en moyenne 140ms par ville :
\begin{center}
	\includegraphics[width=5cm]{timeq7.png}
\end{center}

\section{Index}
En utilisant un index : 
\begin{center}
	\includegraphics[width=5cm]{8jecrois.png}
\end{center}

\section{\$inc et .update}
\begin{center}
	\includegraphics[width=10cm]{Sent/codeQ9.PNG}
	\includegraphics[width=10cm]{Sent/resQ9_1.PNG}
\end{center}
Vérification du résultat
\begin{center}
	\includegraphics[width=10cm]{Sent/resQ9_2.PNG}
\end{center}

\pagebreak
\section{Array dans les documents}
Insert: 
\begin{center}
	\includegraphics[width=6cm]{Sent/insertQ10.PNG}
\end{center}
pull:
\begin{center}
	\includegraphics[width=8cm]{Sent/poule Q10.PNG}
\end{center}
	result pull:
\begin{center}
	\includegraphics[width=8cm]{Sent/res poule Q10.PNG}
\end{center}
	updates:
\begin{center}
	\includegraphics[width=8cm]{Sent/Q10 update.PNG}
	\includegraphics[width=8cm]{Sent/Q10 update 2.PNG}
\end{center}
	results:
\begin{center}
	\includegraphics[width=5cm]{Sent/resUpdateQ10.PNG}
	\includegraphics[width=7cm]{Sent/res 2nd update Q10.PNG}
\end{center}

\section{Liaison de document}
On insert, puis drop et remplace:
\begin{center}
	\includegraphics[width=7cm]{Sent2/Q11 insert users.PNG}
	\includegraphics[width=7cm]{Sent2/insert Q11.PNG}
\end{center}
On vérifie :
\begin{center}
	\includegraphics[width=7cm]{Sent2/Q11 code.PNG}
	\includegraphics[width=7cm]{Sent2/res Q11.PNG}
\end{center}
\pagebreak
En python:
\begin{center}
	\includegraphics[width=10cm]{Sent2/Q11 pymongo code.PNG}
\end{center}

résultat:
\begin{center}
	\includegraphics[width=15cm]{Sent2/Q11 pymongo res.PNG}
\end{center}

\section{Authentification}
Nous avons commencé le TP par protèger la base de donnée. 
Chaque utilisateur (sauf ceux de maxime, qui sont admin) 
n'as les droits que pour sa base. Voici ce qu'on obtient 
dès que quelqu'un se connecte anonymement :

\begin{center}
	\includegraphics[width=10cm]{auth.png}
\end{center}

\section{Map Reducer}

\begin{center}
	\includegraphics[width=5cm]{mapReducer.png}
\end{center}

\section{Backup and Dump}

On fait un backup : 
\begin{lstlisting}
$mongodump -u USERNAMETOHIDE -p PASSWORDTHATISVERYSAFE --out=/tmp
2023-02-05T21:32:32.923+0000	writing admin.system.users to /tmp/admin/system.users.bson
2023-02-05T21:32:32.924+0000	done dumping admin.system.users (7 documents)
2023-02-05T21:32:32.925+0000	writing admin.system.version to /tmp/admin/system.version.bson
2023-02-05T21:32:32.926+0000	done dumping admin.system.version (2 documents)
2023-02-05T21:32:32.929+0000	writing France.communes to /tmp/France/communes.bson
2023-02-05T21:32:32.931+0000	writing France.yes to /tmp/France/yes.bson
2023-02-05T21:32:32.934+0000	writing France.mapReducerOut to /tmp/France/mapReducerOut.bson
2023-02-05T21:32:32.937+0000	writing France.orders to /tmp/France/orders.bson
2023-02-05T21:32:32.950+0000	done dumping France.orders (10 documents)
2023-02-05T21:32:32.959+0000	done dumping France.mapReducerOut (4020 documents)
2023-02-05T21:32:32.959+0000	writing France.map\_reduce\_example to /tmp/France/map\_reduce\_example.bson
2023-02-05T21:32:32.968+0000	done dumping France.yes (4020 documents)
2023-02-05T21:32:32.968+0000	writing mailing.users to /tmp/mailing/users.bson
2023-02-05T21:32:32.971+0000	done dumping France.map\_reduce\_example (4 documents)
2023-02-05T21:32:32.971+0000	writing Analytics.freq to /tmp/Analytics/freq.bson
2023-02-05T21:32:32.971+0000	writing mailing.lists to /tmp/mailing/lists.bson
2023-02-05T21:32:32.972+0000	done dumping mailing.users (3 documents)
2023-02-05T21:32:32.973+0000	done dumping Analytics.freq (1 document)
2023-02-05T21:32:32.974+0000	writing France.map reduce example to /tmp/France/map+reduce+example.bson
2023-02-05T21:32:32.974+0000	done dumping mailing.lists (1 document)
2023-02-05T21:32:32.975+0000	done dumping France.map reduce example (0 documents)
2023-02-05T21:32:33.148+0000	done dumping France.communes (39201 documents)
\end{lstlisting}
On le recharge:

\begin{lstlisting}
$sudo mongorestore -u STILLAGOODUSERNAME -p EPICPASSWORD /tmp/Analytics/freq.bson
2023-02-05T21:38:41.589+0000	WARNING: On some systems, a password provided directly using --password may be visible to system status programs such as `ps` that may be invoked by other users. Consider omitting the password to provide it via stdin, orusing the --config option to specify a configuration file with the password.
2023-02-05T21:38:41.606+0000	checking for collection data in /tmp/Analytics/freq.bson
2023-02-05T21:38:41.606+0000	reading metadata for Analytics.freq from /tmp/Analytics/freq.metadata.json
2023-02-05T21:38:41.607+0000	restoring to existing collection Analytics.freq without dropping
2023-02-05T21:38:41.607+0000	restoring Analytics.freq from /tmp/Analytics/freq.bson
2023-02-05T21:38:41.613+0000	continuing through error: E11000 duplicate keyerror collection: Analytics.freq index: _id_ dup key: { _id: ObjectId('63e008bacc12ed9b5d0c54f8') }
2023-02-05T21:38:41.649+0000	finished restoring Analytics.freq (0 documents, 1 failure)
2023-02-05T21:38:41.649+0000	no indexes to restore for collection Analytics.freq
2023-02-05T21:38:41.649+0000	0 document(s) restored successfully. 
1 document(s) failed to restore.
\end{lstlisting}

Ici, on ne recharge pas, car la base de donnée tourne déjà avec les élements charger, et n'authorise 
pas les duplicat d'id.


\end{document}
